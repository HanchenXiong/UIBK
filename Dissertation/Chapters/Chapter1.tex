% Chapter 1

\chapter{Introduction} % Main chapter title
\label{Chapter1} % For referencing the chapter elsewhere, use \ref{Chapter1} 
\lhead{Chapter 1. \emph{Introduction}} % This is for the header on each page - perhaps a shortened title

\rule{\textwidth}{0.4pt} \\[0.5cm]
\textit{``Before the begining of great brilliance, there must be chaos."}

\begin{flushright}
I Ching
\end{flushright}
\rule{\textwidth}{0.4pt} 
%----------------------------------------------------------------------------------------
\\
This is an introductory chapter which provides an overview of the dissertation. As the title suggests, this dissertation focuses on working with structured data, including 
inference, learning and optimization. In section \ref{sec:motivation}, some motivation and background of studying structured domains are explained. In section \ref{sec:overview}, 
an outline of the dissertation is presented. Basically, the dissertation is composed of three parts and each part is briefly introduced. In section \ref{sec:contribution}, as the author's 
contributions to relevant scientific fields, a list of his research papers during doctoral study is presented.        


\section{Work on Structured Domains}
\label{sec:motivation}
This section mainly explains the concept of "structure" and some structured domains which will be studied later.                  

Generally, structures exist almost everywhere in the universe. For instance, it is usually said that a galaxy, an architecture or a society is a structure.          
Looking closer at these three examples and abstracting their similarities, a rough definition of the "structure" can be derived:     
\begin{definition}
A structure is is a set of elements, which exist with certain states based on the interactions among them. 
\end{definition}

The``interactions" can be also replaced with other terms in different scenarios, \emph{e.g.} dependencies, compatibilities, constraints. 
Two critical points in understanding the definition are: (\emph{i}). there should be more than one element; (\emph{ii}). each element's state can be affected by the states of other ones.  
Based on this specification, 
more "structure" 
examples can also be easily observed. A sentence is a structure by considering words as its elements, an image is a structure as a set of dependent pixels, a human body is a structure composed of 
arms, legs and other parts, just to name a few. Indeed, structured domains are very common in many areas.  


In old times, people are already aware of integrating data collected from different sources to find interesting patterns among them.     
However, this procedure was usually slow and unstable.  
With the development of modern computers and digitization, increasingly more types of data are available in almost all areas.  
There usually exist different kinds of dependencies among multiple data. Therefore, on the one hand, more information can be expected to be extracted  
from more comprehensive data; on the other hand, some extra challenges are also added when confronting multiple dependent data.      
For instance, after pooling multiple data together, working with them obviously goes much beyond classic machine learning techniques for classification 
or regression.
Therefore, it is of significance to study how to conduct different tasks or manipulations on multiple dependent data.          

In this dissertation, multiple data sets are considered as structures, in which different dependencies can 
hence be modeled. Furthermore, structures are encoded within three forms by using: \emph{graphs}, 
\emph{kernels}, and \emph{manifolds} respectively. These three forms match different application domains and are selected 
according to the nature of the task at hand.  

In this dissertation, several structured domains will be considered in following tasks:  
\begin{itemize}
	\item \emph{image segmentation}
	\item \emph{image annotation}
	\item \emph{protein function prediction}
	\item \emph{object action relation modeling}
	\item \emph{3D transformation optimization} 
\end{itemize}
In \emph{image segmentation}, the labels of all pixels (or blocks) within an image construct a structure, in which neighbouring labels should be dependent.  
In \emph{image annotation}, the annotated tags for images also constitute a structured domain, where the presence of each tag can be affected by other ones.       
Similarly, in \emph{protein function prediction}, the functions associated with each protein are also dependent. Actually, image annotation and 
protein functions belong to \emph{multi-label learning}. In \emph{object action relation modeling}, meanwhile, the structure is 
considered in two ways: first, given an object, the actions which can be applied on it and lead to positive effects is a structured domain; second, given an action, 
the objects on which the action can be successfully executed is also a structured domain. In this sense, object-action relation modeling is  
a \emph{two-view learning} paradigm. Finally, in \emph{3D transformation optimization},  3D translation and rotation together is considered  
as a structure. In particular, by representing 3D rotations with $3\times 3$ matrices, there exist hard constraints among the entries of one matrix, \emph{i.e.} they 
are dependent.  



\section{Overview}
\label{sec:overview}
The dissertation can be divided into three parts: \emph{inference}, \emph{learning} and \emph{optimization}. The first part contains Chapter \ref{Chapter2}, where (probabilistic) graphical models 
and corresponding inference are presented. Chapter \ref{Chapter3} and Chapter \ref{Chapter4} belong to the second part, where two principles
for handling structures, \emph{kernels} and \emph{graphs}, and their corresponding learning algorithms are studied respectively. 
The third part contains Chapter \ref{Chapter5}, which studies the optimization on 
\emph{graphs} and \emph{matrix manifolds}.                    

\subsection{Part I: Inference with Graphical Models}
Graphical models are popular tools for probabilistically modeling the interactions among multiple variables. Directed graphical models and undirected graphical models will be introduced, of which 
the representatives are \emph{Bayesian network} and \emph{Markov network} 
respectively. The connections between Bayesian network and Markov networks will be studied.  Markov networks will be more emphasized 
since it provides more flexible modeling capacity.        

Also, three inference algorithms will be presented: \emph{(loopy) belief network}, \emph{variational methods} and \emph{Markov chain Monte Carlo}. 
\subsection{Part II: Structured Output Learning}
The second part will focus on learning. First, Chapter \ref{Chapter3} will continue with Markov networks. In particular, a discriminative case of Markov networks, \emph{conditional random fields} (CRFs) 
will be introduced. However, instead of modeling and inference in Chapter~\ref{Chapter2}, Chapter \ref{Chapter3} is dedicated to investigate learning issues with undirected graphical models.          

Chapter \ref{Chapter4}, by contrast, will treat structured data within \emph{kernel methods}. Kernel is not unfamiliar to machine learning community since it has been incorporated with many 
classic techniques for handling nonlinear patterns. Most previous work only consider kernels on inputs, while in Chapter \ref{Chapter4} kernels on structured output will be particularly 
studied.         


\subsection{Part III: Optimization on Structures}
The third part is for optimization on structured data, which is also quite desirable in many applications. Two types of structure forms will be considered: graphs and matrix manifolds.         
In particular, optimization with graphical models is essentially a generalization of inference (in Chapter \ref{Chapter2}) by considering modes rather than expectations.            
Optimization on matrix manifolds will be explained basically with a practical task: 3D transformation optimization.      


\section{Author's Contribution}
\label{sec:contribution}
During the author's  Phd study, ten relevant research papers were finished, out of which nine are published (or accepted for publication) in 
journals or conference proceedings. They are listed as follows:     
\begin{shaded}
{\Huge I.} \textbf{Hanchen Xiong}, Sandor Szedmak, Justus Piater. {\it Implicit Learning of Simpler Output Kernels for Multi-Lable Prediction}, NIPS workshop on Representation and Learning Methods for Complex Outputs (NIPS-RLCO14).  
\vspace{-.2cm}

{\Huge II.} \textbf{Hanchen Xiong}, Sandor Szedmak, Justus Piater. {\it Towards Maximum Likelihood: Learning Undirected Graphical Models using Persistent Sequential Monte Carlo}, The 6th Asian Conference on Machine Learning (ACML14), pp 205-220, 2014, Journal of Machine Learning Research: Workshop and Conference Proceedings 39, 
\textbf{Best Paper Award}.    
\vspace{-.2cm}

{\Huge III.} \textbf{Hanchen Xiong}, Sandor Szedmak, Justus Piater. {\it Scalable, Accurate Image Annotation with Joint SVMs and Output Kernels}, Neurocomputing Journal (Accepted).  
\vspace{-.2cm}

{\Huge IV.} \textbf{Hanchen Xiong}, Sandor Szedmak, Antonio Rodr{\'i}guez S{\'a}nchez, Justus Piater. {\it Towards Sparsity and Selectivity: Bayesian Learning of Restricted Boltzmann Machine for Early Visual Features},In Proceedings of the 24th International Conference on Artificial Neural Networks (ICANN14), pp 419-426, 2014, Springer LNCS. 
\vspace{-.2cm}

{\Huge V.} \textbf{Hanchen Xiong}, Sandor Szedmak, Justus Piater. {\it Joint SVM for Accurate and Fast Image Tagging},In Proceedings of the 22nd European Symposium on Artificial Neural Network (ESANN14), 
pp 295-330, 2014
\vspace{-.2cm}


{\Huge VI.} \textbf{Hanchen Xiong}, Sandor Szedmak, Justus Piater. {\it 3D Object Class Geometry Modeling with Spatial Latent Dirichlet Markov Random Fields}, In Proceedings of the 35th German Conference on Pattern Recognition (GCPR13), pp 51-60, 2013,  Springer LNCS.  
\vspace{-.2cm}

{\Huge VII.} \textbf{Hanchen Xiong}, Sandor Szedmak, Justus Piater {\it Homogeneity Analysis for Object-Action Relations Reasoning in Kitchen Scenarios}, 
In Proceedings of 2nd Workshop on Machine Learning for Intelligent Systems (MLIS13), pp 37-44,  2013, ACM. 
\vspace{-.2cm}

  {\Huge VIII.} \textbf{Hanchen Xiong}, Sandor Szedmak, Justus Piater {\it A Study of Point Cloud Registration with Probability Product Kernel Functions}, 
In Proceedings of 2013 International Conference on 3D Vision (3DV13), pp 207-214,  2013, IEEE.
\vspace{-.2cm}

 {\Huge IX.} \textbf{Hanchen Xiong}, Sandor Szedmak, Justus Piater {\it Efficient, General Point Cloud Registration With Kernel Feature Maps}, 
In Proceedings of 10th International Conference on Computer and Robot Vision (CRV13), pp 83-90, 2013, IEEE.

\vspace{-.2cm}
\end{shaded}

Besides, one unpublished work (in preparation) is also be presented in the dissertation.   
\begin{shaded}
 {\Huge X.} \textbf{Hanchen Xiong}, Sandor Szedmak, Justus Piater {\it Multi-Label Learning with Kernel Generalized Homogeneity Analysis}, 
Unpublished, 2015.
\end{shaded}
To fit the main theme of the dissertation and minimize content redundancy, only a subset of them (paper I, II, III, VI, VII, IX, X) are inserted at appropriate chapters or 
sections of the dissertation. 
In addition, to keep them self-contained and self-consistent,     
their original content and formats are preserved,   
which should be distinguished from regular paragraphs. The preprints of these publications can be also found at: \url{http://iis.uibk.ac.at/publications}, 
complying with their corresponding copyrights. 
Throughout the dissertation, used papers will be referred to by their corresponding roman numerals specified above. 

